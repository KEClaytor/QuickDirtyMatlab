
\section{Basics}


\subsection{Syntax}

Matlab is case-sensitive, meaning ''X'' is not the same as ''x''.

Every program begins with the command \texttt{program} and ends with \texttt{end program}. You can give these labels, so that an example program might read:
\begin{quote}
\verbatiminput{code01/p01.m}
\end{quote}
\noindent Note: \texttt{!} indicates a comment. They may follow commands


\subsection{Arithemetic}

\begin{quote}
\begin{verbatim}
gfortran p01_program.f08
./a.out
\end{verbatim}
\end{quote}

\noindent The usual gcc flags work (changing the name of the output file):
\begin{quote}
\begin{verbatim}
gfortran p01_program.f08 -o first.out
./first.out
\end{verbatim}
\end{quote}


\subsection{Variables}
Fortran allows for implicit variable types, based on the first character of the variable name. They are the \texttt{real} type if they start with A-H or O-Z, and the \texttt{integer} type if they start with I-N.

It is generally \emph{not recommended} to do this and to declare the variable type instead. To force this include \texttt{implicit none} at the beginning of your code.

You can declare specific variables to be static (unchangable) with the \texttt{parameter} flag.

Arrays may be multi-dimensional with the dimension specified by the number of indicies (each index specifies the length of that dimension).

You can fill an array at once with the \texttt{(/.../)} syntax.
\begin{quote}
\verbatiminput{01_BasicsCode/p02_variables.f08}
\end{quote}


\subsection{Input and Output}
Reading and writing from the screen:
\begin{quote}
\verbatiminput{01_BasicsCode/p03_echo.f08}
\end{quote}
\noindent You will get an error if you try to input a value other than what is expected (eg. real input when integer is expected)

Formated writes
\begin{quote}
\verbatiminput{01_BasicsCode/p04_format.f08}
\end{quote}

\noindent Reading and writing from a file:
\begin{quote}
\verbatiminput{01_BasicsCode/p05_fileio.f08}
\end{quote}
\noindent The file is:
\begin{quote}
\verbatiminput{01_BasicsCode/p05_data.txt}
\end{quote}

\subsection{Code Control}

Loops are handled with do:
\begin{quote}
\verbatiminput{01_BasicsCode/p06_do.f08}
\end{quote}

If statments can be expanded into if else with the usual set of logical operators
 (\texttt{<, >, <=, >=, ==, /=}) and statement combinations
 (\texttt{.and., .or., .not., .eqv., .neqv.}) supported:
\begin{quote}
\verbatiminput{01_BasicsCode/p07_ifelse.f08}
\end{quote}

\noindent There is also some more fine-grain loop control with:

\texttt{exit} will exit a loop

\texttt{cycle} passes up to the next enddo
\begin{quote}
\verbatiminput{01_BasicsCode/p08_doec.f08}
\end{quote}

\noindent You can also use implied do loops with parentheses:
\begin{quote}
\verbatiminput{01_BasicsCode/p09_doi.f08}
\end{quote}
