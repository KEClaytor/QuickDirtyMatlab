
\section{Basics}

\subsection{Arithemetric}
Presumably, you're using Matlab because you would like to do some math.
 Let's start with the basic operations:

\begin{quote}
\verbatiminput{code/ch01_math.m}
\end{quote}
\noindent Note: \texttt{\%} indicates a comment.
 They may follow commands.
 Comments are used to help make code clearer, and more understandable by humans.
 I will use them frequently in the examples to follow.
 You should use them frequently in your code.

\pagebreak
\subsection{Variables}
We can store numbers and letters into variables
 Matlab is case-sensitive, meaning ''X'' is not the same as ''x''.
 Note how we use '';'' to suppress output (or not).

\begin{quote}
\verbatiminput{code/ch01_variables.m}
\end{quote}

\pagebreak
\subsection{Variable Types}
These are the common variable types.

\begin{quote}
\verbatiminput{code/ch01_variabletypes.m}
\end{quote}

\pagebreak
\subsection{Printing to the Screen}
In the example below, we'll print using
 ''disp'',
 ''fprintf'',
 and ''sprintf''.
 Note that the semicolon, '';'', will suppress output to the screen.
 File I/O will come later, focus now on the priting commands.
\begin{quote}
\verbatiminput{code/ch01_print.m}
\end{quote}

\pagebreak
\subsection{Vectors and Matricies}
Great! We've got the basics out of the way.
 Time to dive into what Matlab was made for - vectors and matricies.

\begin{quote}
\verbatiminput{code/ch01_matrix.m}
\end{quote}
