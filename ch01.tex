
\section{Basics}


\subsection{Arithemetric}
Presumably, you're using Matlab because you would like to do some math.
 Let's start with the basic operations:

\begin{quote}
\verbatiminput{code/01_math.m}
\end{quote}
\noindent Note: \texttt{%} indicates a comment.
 They may follow commands.
 Comments are used to help make code clearer, and more understandable by humans.
 I will use them frequently in the examples to follow.
 You should use them frequently in your code.

\subsection{Variables}
We can store numbers and letters into variables
 Matlab is case-sensitive, meaning ''X'' is not the same as ''x''.
 Note how we use '';'' to suppress output (or not).

\begin{quote}
\verbatiminput{code/01_variables.m}
\end{quote}

\subsection{Printing to the Screen}
In the example below, we'll print using
 ''disp'',
 ''fprintf'',
 and ''sprintf''.
 Note that the semicolon, '';'', will suppress output to the screen.
 File I/O will come later, focus now on the priting commands.
\begin{quote}
\verbatiminput{code/01_print.m}
\end{quote}

\subsection{Vectors and Matricies}
Great! We've got the basics out of the way.
 Time to dive into what Matlab was made for - vectors and matricies.

\begin{quote}
\verbatiminput{code/01_matrix.m}
\end{quote}

\subsection{Input and Output}
Reading and writing from the screen:
\begin{quote}
\verbatiminput{01_BasicsCode/p03_echo.f08}
\end{quote}
\noindent You will get an error if you try to input a value other than what is expected (eg. real input when integer is expected)

Formated writes
\begin{quote}
\verbatiminput{01_BasicsCode/p04_format.f08}
\end{quote}

\noindent Reading and writing from a file:
\begin{quote}
\verbatiminput{01_BasicsCode/p05_fileio.f08}
\end{quote}
\noindent The file is:
\begin{quote}
\verbatiminput{01_BasicsCode/p05_data.txt}
\end{quote}

\subsection{Code Control}

Loops are handled with do:
\begin{quote}
\verbatiminput{01_BasicsCode/p06_do.f08}
\end{quote}

If statments can be expanded into if else with the usual set of logical operators
 (\texttt{<, >, <=, >=, ==, /=}) and statement combinations
 (\texttt{.and., .or., .not., .eqv., .neqv.}) supported:
\begin{quote}
\verbatiminput{01_BasicsCode/p07_ifelse.f08}
\end{quote}

\noindent There is also some more fine-grain loop control with:

\texttt{exit} will exit a loop

\texttt{cycle} passes up to the next enddo
\begin{quote}
\verbatiminput{01_BasicsCode/p08_doec.f08}
\end{quote}

\noindent You can also use implied do loops with parentheses:
\begin{quote}
\verbatiminput{01_BasicsCode/p09_doi.f08}
\end{quote}
