
\pagebreak
\section{Image Processing}
If you've been writing perfect Matlab code up to this point... congratulations.
 But eventually you're likely to encounter a bug in you code.
 The graphical nature of the editor, and the interpreted nature of the code allows for easy debugging.

\subsection{Image Read and Write}

\begin{quote}
\verbatiminput{code/ch07_imagereadwrite.m}
\end{quote}

\pagebreak
\subsection{Displaying Images}
There are three frequent commands you'll use when displaying images.

\emph{imshow} is the quick, catch-all image displayer.
 It displays the image at a 1:1 pixel resolution on your monitor.
 It deals with most of the colormaps and data types for you.

\emph{image} displays the image stretched to fit the axes.
 Both imshow and image suffer from a [0,1] cutoff in data.
 If the data exceed these bounds (lower or upper) it will be colored the same as the bound.

\emph{imagesc} stretches the image to the current axes, but also scales it so the minimum and maximum values are at the edges of the colorbar.
 Because it scales the image, you'll always see all of the data unlinke with image.
 However, a single very large value can throw the colorbar off.

\begin{quote}
 \verbatiminput{code/ch07_display.m}
\end{quote}

\pagebreak
\subsection{Colorbars and Colormaps}


\begin{quote}
 \verbatiminput{code/ch07_colorbars.m}
\end{quote}
