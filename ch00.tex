
\pagebreak
\section{Introduction}

Thanks for picking up \emph{The Quick and Dirty Guide to Matlab}. 
The goal of this series is not to give you a complete overview, or even a throuough tutorial to a language,
 but rather to give you the basics and give it to you fast enoguh so that you can get started programming in a day or two. 
You'll also find a links and resources section at the end which consists of the resources I found useful when compiling this guide.

Matlab has grown over time and now includes the base Matlab install and also many 'toolboxes' which include more specialized functionality.
 As such, it has become quite an expensive product, and, if your license is not paid for you, can be expensive to own.
 Fear not, there are legal and open-source alternatives.
 If you want nearly an identical Matlab clone try (TODO: Link) GNU Octave.
 You can also configure (TODO: Link) Python with (TODO:) NumPy and (TODO) SciPy, MatPlotLib, and iPython.
 That may sound like quite the list. You can have a look at my (TODO) Quick and Dirty Guide to Python. For help in setting this up.

All example code in this book may be downloaded and run. The examples contain what I consider to be the minimum through unit for a certain concept.
You may find all the examples for download at either:
 \href{http://people.duke.edu/~kec30/}{my website}, the book's
 \href{https://github.com/KEClaytor/QuickDirtyMatlab}{github page} (where you can also get the TeX source for this document),
 or simply copy and paste them into the editor.

You're welcome to skip to the next page to get going, but if you want I will say a little about the language here.
MATLAB was written in NNNN, initially as a means of teaching undergraduates linear algebra (MATLAB = MATrix LABoratory).

Matlab code is an interpreted language, which means that is is not quite as fast as compiled languages such as C or Fortran.
 The interpreters improve with every iteration, however, and interpreted Matlab is reasonably fast now.
 There are some optimizations that help improve performance, and I'll try to point them out when they arise.

Good luck with your forrays into Matlab, I hope you find the rest of this guide useful.

    --Kevin Claytor
