
\section{Code Control}

\subsection{For}
A basic instruction is to tell the computer to do something a given number of times.
 The ''for'' loop is usually used for this.

\begin{quote}
\verbatiminput{code/chch02_for.m}
\end{quote}
\noindent

\pagebreak
\subsection{While}
The while loop can continue incrementing until a given condition is met.
 This can be useful in iterating until you reach a tolerance.
 But can also continue forever if you are not careful.
 If this happens, use ''ctrl-C'' to stop the currently running code.

\begin{quote}
\verbatiminput{code/chch02_while.m}
\end{quote}

\pagebreak
\subsection{If/Else/Elseif}
If you need to switch between two or more conditions, if / else should be used.

\begin{quote}
\verbatiminput{code/ch02_ifelse.m}
\end{quote}

\pagebreak
\subsection{Switch}
The switch statement is useful for switching between multiple conditions.
 It is equivalent to a string of if/elseif.
 After one case of the switch statement is executed, the program continues \emph{after} the swtich statement.
 This is notably different from C.

\begin{quote}
\verbatiminput{code/ch02_switch.m}
\end{quote}

\pagebreak
\subsection{Vectorization}
Matlab is interpreted, meaning that loops are not always optimized.
 An easy way of getting more performance is to vectorize them.
 Matlab can handle vector operations (remember it was made for matrix / vector math) \emph{much} faster than for loops.

You can also use logical indexing to handle some if/then/else statements.
 I find this often makes the code more readable as well.
 
\begin{quote}
\verbatiminput{code/ch02_vectorization.m}
\end{quote}
