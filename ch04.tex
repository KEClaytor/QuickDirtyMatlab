
\section{Plotting}
One of the advantages of Matlab is it's excellent plotting and support for many different plotting types.
 I'll start by demonstrating the basics of how to plot.
 If you're interested in a specific type of plot I would recommend the help files.
 After the basic plots I'll show you more efficient plot updating and customization with the use of handles.

\subsection{2D Plots}
The demonstrations below show a scattering of 2D plots and their output.
 You can find many more in Matlab's help secion on plots.

\begin{quote}
\verbatiminput{code/04_plot2d.m}
\end{quote}

\pagebreak
\subsection{3D Plots}
Here are some demonstration 3D plots.
 Again these are just the basics many more examples of 3D plots can be found in the Matlab help files.

\begin{quote}
 \verbatiminput{code/04_plot3d.m}
\end{quote}

\pagebreak
\subsection{Subplots}
Subplots allow you to fit multiple axes within a single figure.

\begin{quote}
 \verbatiminput{code/04_subplots.m}
\end{quote}

\pagebreak
\subsection{Plot Handles (get and set)}
''Handles'' are a way we can use to get at the underlying components of a plot.
 The ''get'' command returns the properties of the current handle.
 And the ''set'' command allows you to change those properties.

\begin{quote}
\verbatiminput{code/04_handles.m}
\end{quote}

\pagebreak
\subsection{Labels, Fonts, and Defaults}
I'm a big believer that \emph{every plot should be labeled}.
 Despite not having done so previously (see the exercies), I'll now show you how to do this.

\begin{quote}
\verbatiminput{code/04_labels.m}
\end{quote}

If you are finding yourself frequently setting the fontsize or weight on an axes, a better alternative is to change the axes default.
 This process is a bit more involved.
 Start by making a file called ''startup.m'' in your matlab base directory, eg; ''C:\Users\KEClaytor\Documents\Matlab''.
 We'll set the properties for the default axes and figure handle (0).
 Below is my startup.m file.
 
\noindent You'll notice that I also override the default colorscale (mine has black in the middle).
 If you choose to do this, make sure that the colorscale file appears on the path (preferably with startup.m).

\pagebreak
\subsection{Plot Legends}
The basic plot legends are simple and easy to use.
 Things get more complicated when you start layering components and do not want a legend for all of them.
 We'll start having to use handles and adjust properties using them.

\begin{quote}
\verbatiminput{code/04_legends.m}
\end{quote}
