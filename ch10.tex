
\pagebreak
\section{Parallelization}
As you take on larger and more computationally intensive projects, you code can take longer to run.
 To improve your code's run time you can take advantage of parallelization options.
 While this guide has been largely toolbox-free, you'll need the parallel computing toolbox for this chapter.

\subsection{Parfor}
Dual, quad, or even higer CPU cores are found in modern computers.
 You can take advantage of these CPU's by using a parfor loop.
 Parfor (parallel for) directs matlab to different steps of the loop on your cores.
 Since the different iterations are run on separate cores, your loop should be desiged accordingly.
 Make sure to not have one loop depend on a previous loop's result.
 And data needs to be constructed in a manner so that each core can recieve its part.
 This data 'slicing' is the cause of most headaches when it comes to parallelization.

\begin{quote}
\verbatiminput{code/ch10_parfor.m}
\end{quote}

\pagebreak
\subsection{GPU Computation}
The GPU has a different architecture than the CPU.
 Instead of trying to do one computation as fast as possible, they do slower computations but many more of them at once (256-1024+).
 Matlab is integrating tightly with the NVidia cards, but options also exist for ATI cards.

The GPU is not always a first-choice for speeding up code.
 You'll need to transfer data to and from the graphics card, which will impose additional overhead time.
 Typically graphics cards have less memory than your system, so you may run out.

\begin{quote}
 \verbatiminput{code/ch10_gpu.m}
\end{quote}
