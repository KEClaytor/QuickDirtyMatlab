
\pagebreak
\section{Resources}
As you've now seen, this is only the roughest introduction to the Matlab.
 You'll likely need more material to supplement your learning, and for more of the subtltly that could not be expressed here.
 These are the resouces I most frequently visit when using Matlab and those that I recall using when learning matlab.

\begin{itemize}
 \item Type in ''help command'' to get a quick description and usage syntax of "command".

 \item Likewise ''doc command'' will bring up the the command in the help browser, often with example images and code.
 This is also availbe as \href{http://www.mathworks.com/help/matlab/}{online documentation}.

 \item Matlab runs a question and answer syle site called \href{http://www.mathworks.com/matlabcentral/answers/}{Matlab Answers}

 \item There is a great third-party alternative in \href{http://stackoverflow.com/questions/tagged/matlab}{Stack Overflow}

 \item The \href{http://www.mathworks.com/matlabcentral/fileexchange/}{Matlab File Exchange (FEX)} is a great site for finding code.
 If you don't know where to start, I would recommend browsing the \href{http://blogs.mathworks.com/pick/}{Pick of the Week}.
 Seeing how other people code is also sure to give you ideas on what / how to do things.

 \item You can also find code on \href{https://github.com/}{Github}.

 \item Here is a more in-depth Mathworks overview of building GUIs: \href{http://www.mathworks.com/help/pdf_doc/matlab/buildgui.pdf}{PDF GUI Overview}

 \item If you get really into GPU programs, there are some other thrid-party alternatives that work with Matlab.
\end{itemize}
